\documentclass[tikz,border=2pt]{standalone}
\usepackage{tikz}
\usetikzlibrary{arrows}
\usetikzlibrary{3d}
\usepackage{cmap}
\usepackage[T2A]{fontenc}
\usepackage[utf8x]{inputenc}
\usepackage[english, russian]{babel}
\usetikzlibrary
    {
        decorations.pathmorphing,
        decorations.pathreplacing,
        decorations.markings,
        shapes.misc,
        patterns,
        calc,
        scopes,
        arrows,
        fadings,
        through,
        shapes.misc,
        arrows.meta,
        3d,
        quotes,
        angles,
        babel
    }
\newcommand\irregularcircle[2]{% radius, irregularity
  let \n1 = {(#1)+rand*(#2)} in
  +(0:\n1)
  \foreach \a in {10,20,...,350}{
    let \n1 = {(#1)+rand*(#2)} in
    -- +(\a:\n1)
  } -- cycle
}

\newcommand\irregular[2]{% radius, irregularity
  let \n1 = {(#1)+rand*(#2)} in
   (0:\n1)
  \foreach \a in {10,20,...,350}{
    let \n1 = {(#1)+rand*(#2)} in
    (\a:\n1)
    % -- +(\a:\n1)
  }% -- cycle
}


\begin{document}
\begin{tikzpicture}[scale=1.5,
    %Option for nice arrows
    >=stealth, %
    inner sep=0pt, outer sep=2pt,%
    axis/.style={thick,->},
    wave/.style={thick,color=#1,smooth},
    polaroid/.style={fill=black!60!white, opacity=0.3},
    interface1/.style={draw=gray!60,
        % The border decoration is a path replacing decorator. 
        % For the interface style we want to draw the original path.
        % The postaction option is therefore used to ensure that the
        % border decoration is drawn *after* the original path.
        postaction={draw=gray!60,decorate,decoration={border,angle=-135,
                    amplitude=0.2cm,segment length=0.5mm}}}, 
]
    \draw[very thick] (0,0) node[left] {} -- ++(3,0) node[right] {$W_v$};
    \draw[very thick] (0,2) node[left] {} -- ++(3,0) node[right] {$W_c$};

    \draw[] (2,1) node[left,align=center] {Глубокий\\уровень} -- ++(1,0) ;
    \draw[] (0,0.33) node[left,align=center] {Мелкий\\уровень} -- ++(1,0) ;
    \draw[] (0,2-0.33) node[left,align=center] {Мелкий\\уровень} -- ++(1,0) ;
    % \foreach \dy in {0.1,0.2,...,0.6}{
    %     \draw[] (0,2+\dy) -- ++(2,0);
    %     \draw[] (0,0-\dy) -- ++(2,0);
    % }

    % \draw[-latex] (0.25,-0.3) -- node[sloped,above] {Генерация} (0.25,2.3);
    \draw[-latex] (2.33,0) -- (2.33,1);
    \draw[-latex] (2.66,1) -- (2.66,0);
    \draw[-latex] (2.66,2) -- (2.66,1);
    \draw[-latex] (2.33,1) -- (2.33,2);

    \draw[latex-] (0.25,0) -- ++(0,0.33);
    \draw[-latex] (0.75,0) -- ++(0,0.33);


    \draw[latex-] (0.25,2) -- ++(0,-0.33);
    \draw[-latex] (0.75,2) -- ++(0,-0.33);

    \draw[latex-, densely dashed] (0.4,0.33) -- ++(0,2-0.33-0.33);
    \draw[-latex, densely dashed] (0.6,0.33) -- ++(0,2-0.33-0.33);
    % \draw[latex-] (2-0.25,-0.3) -- node[sloped,above] {Рекомбинация} (2-0.25,2.3);

    % \draw (1,2.6) node[above=0.2em,align=center] {Разрешенные состояния \\ в зоне проводимости};
    % \draw (1,-0.6) node[below=0.2em,align=center] {Разрешенные состояния \\ в валентной зоне};


\end{tikzpicture}
\end{document}